%% !TeX spellcheck = de_DE
%% spellcheck-language "de"
% \begin{abstract}

Semantische Heterogenität stellt nach wie vor ein grundlegendes Hindernis für die Datenintegration dar: Unabhängig erstellte Tabellen kodieren äquivalente Konzepte mit inkompatiblen Schemata, Spaltenüberschriften und Namenskonventionen. Semantische Annotation, d.h. die Abbildung tabellarischer Daten auf eine kontrollierte Ontologie, ist essenziell für die Transformation isolierter tabellarischer Artefakte in interoperable, abfragbare Ressourcen. Traditionelle Ansätze, die auf Regeln, Zeichenkettenabgleich oder überwachtem maschinellem Lernen basieren, leiden unter mangelnder Flexibilität, begrenzter Generalisierungsfähigkeit und der Abhängigkeit von kostenintensiven annotierten Daten. Large Language Models (LLMs) bieten aufgrund ihres Weltwissens und ihrer Zero-Shot-Inferenzfähigkeiten eine vielversprechende Alternative. Die naive Anwendung von LLMs auf semantische Annotation birgt jedoch kritische Herausforderungen: Halluzination nicht existierender Ontologieklassen, inkonsistente Ausgaben bei wiederholten Anfragen sowie fehlende strukturelle Garantien bezüglich der Zielontologie. Diese Arbeit präsentiert den Entwurf, die Implementierung und die Evaluation einer End-to-End-Plattform für ontologiegetriebene semantische Annotation, die diese Herausforderungen adressiert. Die Plattform transformiert LLMs von offenen Generatoren zu eingeschränkten Entscheidungsmaschinen, die eine kontrollierte Navigation über eine als gerichteter azyklischer Graph (DAG) modellierte Ontologie durchführen. Eine Breitensuche (BFS) leitet den Annotationsprozess von groben zu feingranularen Konzepten, während Chain-of-Thought (CoT) Prompting explizites Schlussfolgern bei jedem Entscheidungsschritt fördert. Ein Ensemble Decision Making (EDM) Mechanismus aggregiert Urteile mehrerer LLM-Agenten, um Varianz zu reduzieren und Robustheit zu verbessern. Die Plattform und der Algorithmus werden auf einem realen Benchmark aus dem Energiebereich evaluiert, der 47 Tabellen mit 431 Spalten sowie eine Ontologie (Building Energy Ontology) mit 602 Klassen umfasst. Die beste Konfiguration erreicht einen pfadbasierten F$_1$-Wert von 38,16\%, wobei CoT-Prompting eine Verbesserung von 6,35 Prozentpunkten gegenüber direktem Prompting erzielt. Eine systematische Kosten-Genauigkeits-Analyse identifiziert Pareto-effiziente Konfigurationen und liefert Einsatzrichtlinien für verschiedene betriebliche Anforderungen. Die Hauptbeiträge umfassen eine modulare Vier-Schichten-Architektur, eine Multi-Provider-LLM-Abstraktion, die sowohl kommerzielle APIs als auch lokale Modelle unterstützt, sowie umfassende Nachvollziehbarkeit für reproduzierbare Experimente.

% \end{abstract}
